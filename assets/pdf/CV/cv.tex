\documentclass[11pt,letterpaper]{article}

\usepackage{slantsc}
\usepackage{amsfonts}
\usepackage{amsmath}
\usepackage{charter}
\usepackage{enumitem}
\usepackage{mathrsfs}
\usepackage{pifont}
\usepackage{bbold}
\usepackage[dvipsnames]{xcolor}
\usepackage[T1]{fontenc}
\usepackage{textcomp}
\usepackage{fancyhdr}
\usepackage{dashrule}
\usepackage{times}
\usepackage[square, comma, sort&compress, numbers]{natbib}
\usepackage[pdftex]{hyperref}
\usepackage[utf8]{inputenc}
\usepackage{color}

\newcommand{\allnotes}[1]{\textit{#1}} 
\newcommand{\alvin}[1]{\allnotes{\textcolor{purple}{[Alvin: #1]}}}
\newcommand{\ryan}[1]{\textcolor{blue}{#1}}
\setlength{\paperwidth}{8.5in}
\setlength{\hoffset}{-9.7mm}
\setlength{\oddsidemargin}{0mm}
\setlength{\textwidth}{184.3mm}
\setlength{\columnsep}{6.3mm}
\setlength{\marginparsep}{0mm}
\setlength{\marginparwidth}{0mm}  
\setlength{\paperheight}{11in}
\setlength{\voffset}{-8.4mm}
\setlength{\topmargin}{-9mm}
\setlength{\headheight}{0mm}
\setlength{\headsep}{3mm}
\setlength{\topskip}{10mm}
\setlength{\textheight}{260.2mm}
\setlength{\footskip}{12.4mm}
\setlength{\parindent}{1pc}
\linespread{1}
    
\lfoot{}
\cfoot{}
\rfoot{}
\hypersetup{colorlinks,%
linkcolor=black,%
urlcolor=[rgb]{0,0,0.6},%
pdftex}

\newcommand{\myline}{\par
  \kern3pt % space above the rules
  \hrule height 0.8pt
  \kern5pt % space below the rules
}

\newenvironment{ssubtitle}{\fontfamily{ptm}\selectfont}{}
\newcommand{\subtitle}[1]{\Large{\textsc{#1}}}
\newenvironment{headd}{\fontfamily{put}\selectfont}{}

\begin{document}\pagestyle{empty}
\begin{center}
\begin{headd}{\huge\textbf{\textrm{YuRun Yuan}}}\end{headd}\\
\normalsize
\vspace{3pt}
\ding{38} (+86)~152 1221 6859$|$~\ding{41} yr\_yuan@mail.ustc.edu.cn
$|$ Homepage: \href{https://ryanyuan-yyr.github.io/}{ryanyuan-yyr.github.io}
\end{center}

\noindent
\begin{ssubtitle}\subtitle{Education}\end{ssubtitle}
\myline

\noindent
{\textbf{University of Science and Technology of China (USTC)} \hfill Hefei, China}\\
\textit{B.S. in Computer Science with honors} \hfill Sep 2019 -- Jun 2023(expected) 
\begin{itemize}[noitemsep, topsep=0pt, leftmargin=11pt]
\item GPA: 3.97/4.3 (ranking 3/256)
\item Core courses: Principles and Techniques of Compiler (100/100), Foundations of Algorithms (97/100), Operating Systems (Honors) (97/100), Computer Organization \& Design (94/100), Data Structures (95/100)
\end{itemize}

\vspace{8pt}\noindent
\begin{ssubtitle}\subtitle{Selected Research Projects}\end{ssubtitle}
\myline

\noindent
\textbf{Manage Heterogeneous Memory Hierarchy with Java Runtime} \hfill Feb 2022 - Present\\ 
\textit{Research Intern, Advisor:} \href{http://jianh.web.engr.illinois.edu/}{\textit{Prof. Jian Huang}} \hfill {\href{http://jianh.web.engr.illinois.edu/platformx/}{PlatformX Group, UIUC}}
\begin{itemize}[noitemsep, topsep=0pt, leftmargin=11pt]
        \item Identified a certain object access pattern in JVM to locate frequently accessed objects
        \item Proposed a new policy that leverages the semantic information in JVM to place objects with different access frequencies in different memories, taking advantage of low latency of fast memory and high capacity of slow memory
        % \item \alvin{What else? Like some evaluation metrics}
\end{itemize}

\noindent
\textbf{Extension of a DSL compiler for ML applications} \hfill Nov 2021 - Jan 2022\\
\textit{Research Assistant, Advisor:} \href{http://staff.ustc.edu.cn/~chengli7/}{\textit{Prof. Cheng Li}} \hfill {\href{http://adsl.ustc.edu.cn/}{Advanced Data Systems Laboratory, USTC}}
\begin{itemize}[noitemsep, topsep=0pt, leftmargin=11pt]
    \item Studied popular gradient compression algorithms used in DNN training, including TernGrad and 3-LC
    % \item Extended \texttt{CompLL}, a compiler that aims to help practitioners develop gradient compression algorithms by converting a C-like DSL code into highly-optimized CUDA C++ code, to support more language features, including loops, arrays, and an operator \texttt{prefix\_sum}.
    \item Extended \texttt{CompLL}, a compiler that translates C-like DSL code into highly-optimized CUDA C++ code, to support more language features, including loops, arrays, and a new operator \texttt{prefix\_sum} % \alvin{Is this compiler designed only for gradient compression algorithms? If not, don't add it.}\ryan{[This compiler is actually designed only for gradient compression algorithms]}
    \item Implemented the 3-LC algorithm in DSL with only 69 lines of code, which greatly reduces developers' burden
\end{itemize}

\vspace{8pt} \noindent
\begin{ssubtitle}\subtitle{Selected Course Projects}\end{ssubtitle}
\myline

\noindent
\textbf{A Compiler for \texttt{cminusf}, a C-like Programming Language} \hfill Oct 2021 - Jan 2022 \\
\textit{Principles of Compiler Design} \hfill {\href{https://github.com/ryanyuan-yyr/2021fall-compiler\_cminus}{Project Link}}
%{\href{https://github.com/ryanyuan-yyr/2021fall-compiler\_cminus}{github.com/ryanyuan-yyr/2021fall-compiler\_cminus}}
\begin{itemize}[noitemsep, topsep=0pt, leftmargin=11pt]
    \item Extended the syntax of \texttt{cminusf} to support structures, class templates, member functions, and operator overloading
    \item Implemented the lexical and semantic analysis of the compiler with \texttt{lexer} and \texttt{bison}
    \item Led a group of 3 to build the abstract syntax tree, translate \texttt{cminusf} source code into intermediate representation, and implement machine-independent optimization(constant propagation, live-variable analysis, and loop invariant hoist) with LLVM
\end{itemize}

\noindent
\textbf{A Primary 5-Stage-Pipelined RISC-V CPU with FPGA} \hfill May 2021 - Jun 2021 \\
\textit{Computer Organization \& Design} \hfill {\href{https://github.com/ryanyuan-yyr/COD-Labs}{Project Link}}
%{\href{https://github.com/ryanyuan-yyr/COD-Labs}{github.com/ryanyuan-yyr/COD-Labs}}
\begin{itemize}[noitemsep, topsep=0pt, leftmargin=11pt]
        \item Implemented a pipelined CPU in \texttt{Verilog}, supporting arithmetic instructions, load/store instructions, memory mapped I/O, instruction \texttt{ecall} and interrupts
        \item Evaluated the CPU performance in the Vivado simulator and on a real FPGA board
\end{itemize}

\vspace{8pt} \noindent
\begin{ssubtitle}\subtitle{Skills \& Techniques}\end{ssubtitle}
\myline
\begin{itemize}[noitemsep, topsep=0pt, leftmargin=11pt]
    \item \textit{TOEFL:} Reading 30, Listening 24, Speaking 22, Writing 29
    \item \textit{GRE:} Verbal 152, Quantitative 170, Analytical Writing 4.0
    \item \textit{Programming Skills:} C/C++, Python, Java, Rust, Verilog, SQL
    \item \textit{Software Skills:} Vivado, gem5, \LaTeX, Maxine JVM
\end{itemize}

\vspace{8pt} \noindent
\begin{ssubtitle}\subtitle{Awards \& Honors}\end{ssubtitle}
\myline
\begin{itemize}[noitemsep, topsep=0pt, leftmargin=11pt]
    \item Shenzhen Stock Exchange Scholarship \hfill 2021
    \item Scholarship for Outstanding Students (Gold award, 15/172) \hfill 2021
    \item Hua Xia Talent Program First Prize Scholarship (12 / 254 students major in CS) \hfill 2020-2021
\end{itemize}


\end{document}
