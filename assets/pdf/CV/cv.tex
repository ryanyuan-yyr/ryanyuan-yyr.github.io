%%%%%%%%%%%%%%%%%%%%%%%%%%%%%%%%%%%%%%%%%
% Medium Length Professional CV
% LaTeX Template
% Version 2.0 (8/5/13)
%
% This template has been downloaded from:
% http://www.LaTeXTemplates.com
%
% Original author:
% Trey Hunner (http://www.treyhunner.com/)
%
% Important note:
% This template requires the resume.cls file to be in the same directory as the
% .tex file. The resume.cls file provides the resume style used for structuring the
% document.
%
%%%%%%%%%%%%%%%%%%%%%%%%%%%%%%%%%%%%%%%%%

%----------------------------------------------------------------------------------------
%	PACKAGES AND OTHER DOCUMENT CONFIGURATIONS
%----------------------------------------------------------------------------------------

\documentclass{resume} % Use the custom resume.cls style

\usepackage[left=0.75in,top=0.6in,right=0.75in,bottom=0.6in]{geometry} % Document margins

\usepackage{hyperref}

\name{Yurun Yuan} % Your name
\address{Western Campus, University of Science and Technology of China \\ Anhui, China} % Your address

\address{(+86)~152~$\cdot$~1221~$\cdot$~6859 \\ yr\_yuan@mail.ustc.edu.cn} % Your phone number and email

\address{\href{https://ryanyuan-yyr.github.io/}{ryanyuan-yyr.github.io} \\ \href{https://github.com/ryanyuan-yyr}{github.com/ryanyuan-yyr}}

\begin{document}

%----------------------------------------------------------------------------------------
%	EDUCATION SECTION
%----------------------------------------------------------------------------------------

\begin{rSection}{Education}

    {\bf University of Science and Technology of China (USTC)} \hfill {September 2019 - Present} \\
    Major in Computer Science \\
    GPA: 3.97/4.3 (ranking 6/252), \ Average score: 92.24/100 \\
    Core courses: \\
    \begin{tabular}{l@{\hspace{1ex}} @{} >{}l @{\hspace{6ex}} l l@{\hspace{1ex}} >{}l @{\hspace{6ex}} l }
        $\cdot$ & Operating Systems (Honors)        & 97/100 &
        $\cdot$ & Computer Organization \& Design   & 94/100   \\
        $\cdot$ & Data Structures                   & 95/100 &
        $\cdot$ & Introduction to Computing Systems & 95/100   \\
        $\cdot$ & Programming (Advanced)            & 98/100 &
        $\cdot$ & Graph Theory                      & 97/100
    \end{tabular}
\end{rSection}

%----------------------------------------------------------------------------------------
%	WORK EXPERIENCE SECTION
%----------------------------------------------------------------------------------------

\begin{rSection}{Research Experience}

    \begin{rSubsection}{\href{http://adsl.ustc.edu.cn/}{Advanced Data Systems Laboratory, USTC}}{November 2021 - Present}{Research Assistant, Advisor: \href{http://staff.ustc.edu.cn/~chengli7/}{Cheng Li}}{\href{https://gitlab.com/hipress/hipress}{gitlab.com/hipress/hipress}}

        \item Researched on gradient compression algorithms used in DNN training, including TernGrad and 3-LC.
        \item Extended \texttt{CompLL}, a compiler that aims to help practitioners develop gradient compression algorithms by converting a DSL into highly-optimized CUDA C++ code, to support more language features, including loops, arrays, and an operator \texttt{prefix\_sum}.
        \item Implemented algorithm 3-LC in the DSL and converted it into CUDA C++ code using the extended compiler.
    \end{rSubsection}

    %------------------------------------------------

    \begin{rSubsection}{\href{https://luna.bdaa.pro/}{Team LUNA}, \href{http://bigdata.ustc.edu.cn/}{Big Data Laboratory, USTC}}{October 2021 - December 2021}{Backend Developer}{\href{https://gitlab.com/ryanyuan-yyr/luna-ailab-api}{gitlab.com/ryanyuan-yyr/luna-ailab-api}}
        \item Studied the usage of MySQL, ORM and Flask.
        \item Built the backend for the API website, which manages users' data in MySQL databases and provides user authentication services.
    \end{rSubsection}
\end{rSection}
%----------------------------------------------------------------------------------------
%	TECHNICAL STRENGTHS SECTION
%----------------------------------------------------------------------------------------

\begin{rSection}{Selected Course Projects}
    \begin{rSubsection}{A Compiler for \texttt{cminiusf}, a C-like Programming Language}{October 2021 - January 2022}{Compilers Principles, prof.\href{http://staff.ustc.edu.cn/~chengli7/}{Cheng Li}}{\href{https://github.com/ryanyuan-yyr/2021fall-compiler\_cminus}{github.com/ryanyuan-yyr/2021fall-compiler\_cminus}}
        \item Used \texttt{lexer} and \texttt{bison} to implement the lexical and semantic analysis.
        \item Led a group of 3 to build the abstract syntax tree, translate \texttt{cminusf} source code into LLVM IR, and  implement machine independent optimization(constant propagation, live-variable analysis and loop invariant hoist) with C++.
        \item Extended the syntax of \texttt{cminusf}, including structures, member functions, class templates and operator overloading.
    \end{rSubsection}

    \begin{rSubsection}{A Primary 5-Stage-Pipelined RISC-V CPU with FPGA}{May 2021 - June 2021}{Computer Organization \& Design, prof.\href{http://staff.ustc.edu.cn/~cswang/}{Chao Wang}}{\href{https://github.com/ryanyuan-yyr/COD-Labs}{github.com/ryanyuan-yyr/COD-Labs}}
        \item Implemented a pipelined CPU in \texttt{Verilog}, supporting arithmetic instructions, load/store instructions, memory mapped I/O, instruction \texttt{ecall} and interrupts.
        \item Performed stimulation and synthesis on Vivado. Programmed on the FPGA development board.
    \end{rSubsection}

    \begin{rSubsection}{Shell in Rust}{May 2021}{Operating Systems(Honors), prof.\href{http://staff.ustc.edu.cn/~kxing/}{Kai Xing}}{\href{https://github.com/ryanyuan-yyr/OSH-2021-Labs/tree/main/lab2}{github.com/ryanyuan-yyr/OSH-2021-Labs/tree/main/lab2}}
        \item Learned about basic concepts and syntax of Rust.
        \item Implemented a shell in Rust, supporting pipe, \texttt{SIGINT}, and I/O redirection.
    \end{rSubsection}

    \begin{rSubsection}{FAT16 in User Space}{June 2021}{Operating Systems, prof.\href{http://staff.ustc.edu.cn/~ykli/}{Yongkun Li}}{\href{https://github.com/ryanyuan-yyr/FAT-filesystem}{github.com/ryanyuan-yyr/FAT-filesystem}}
        \item Used FUSE to implement a FAT16 filesystem on Ubuntu, supporting the creation and removal of files and directories.
    \end{rSubsection}

\end{rSection}

\begin{rSection}{Miscellaneous}
    \begin{tabular}{ @{} >{\bfseries}l @{\hspace{6ex}} l }
        Language           & TOEFL: 105/120. Reading 30, Listening 24, Speaking 22, Writing 29      \\
        Awards             & Scholarship for Outstanding Students, 2020 - 2021 (Gold award, 15/172) \\
        Programming skills & C, C++, Python, Rust, Verilog                                          \\
        Software skills    & Vivado, \LaTeX, Visual Studio Code                                     \\
        Volunteer work     & Mathematics tutor for high school students
    \end{tabular}
\end{rSection}

\end{document}
